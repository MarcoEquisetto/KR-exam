\documentclass[a4paper,11pt]{article}
\usepackage[utf8]{inputenc}
\usepackage[T1]{fontenc}
\usepackage{geometry}
\usepackage{titlesec}
\usepackage{graphicx}
\usepackage{listings}
\usepackage{xcolor}
\usepackage{hyperref}
\usepackage{booktabs}
\usepackage{float}
\usepackage{caption}
\usepackage[margin=2.5cm]{geometry}

% Page Layout
\geometry{left=2.5cm, right=2.5cm, top=2.5cm, bottom=2.5cm}

% Professional Header Formatting
\titleformat{\section}{\Large\bfseries\color{black}}{\thesection}{1em}{}
\titleformat{\subsection}{\large\bfseries\color{black}}{\thesubsection}{1em}{}

% Code Listing Style
\definecolor{codegray}{rgb}{0.95,0.95,0.95}
\definecolor{codeblue}{rgb}{0.0,0.0,0.6}
\lstset{
    basicstyle=\ttfamily\small,
    keywordstyle=\color{blue},
    commentstyle=\color{gray},
    stringstyle=\color{red},
    breaklines=true,
    numbers=left,
    numberstyle=\tiny,
    frame=single,
    tabsize=4,
    captionpos=b
}

\hypersetup{
    colorlinks=true,
    linkcolor=blue,
    citecolor=blue,
    urlcolor=blue,
    pdftitle={Implementation of a Solver for the Union of Theories of Equality, Lists, and Arrays},
    pdfauthor={Marco Equisetto}
}

% Title Page Data
\title{
    \vspace{2cm}
    \textbf{\huge Clinical Healthcare Ontology} \\
    \vspace{0.5cm}
    \Large Knowledge Representation
    \vspace{1.5cm}
}
\author{\textbf{Equisetto Marco} \\ University of Verona / Department of Computer Science \\Artificial Intelligence}
\date{\today}

\begin{document}

\maketitle
\thispagestyle{empty}
\newpage

% -------------------------------------------------------------------
% Abstract
% -------------------------------------------------------------------
\begin{abstract}
\noindent This report documents the design, implementation, and validation of a semantic ontology for a Clinical Healthcare domain. Developed using OWL 2 DL (Web Ontology Language), the system models complex hospital environments, including staff hierarchies, patient care workflows, and surgical requirements. 

Key features include automated inference of patient criticality, strict cardinality constraints for surgical staffing (e.g., the "Two-Nurse Rule"), and the use of closure axioms to validate resource allocation. The ontology has been successfully tested using the HermiT and Pellet reasoners to verify consistency and identify logical contradictions in non-compliant data scenarios.
\end{abstract}

\tableofcontents

% -------------------------------------------------------------------
% Section 1: Introduction
% -------------------------------------------------------------------
\section{Introduction}
Modern healthcare systems require robust data modeling to ensure patient safety and operational efficiency. This project aims to create a formal ontology that acts not just as a database schema, but as an intelligent logic layer capable of validating clinical events.

The ontology was developed in \textbf{Protégé 5.6} and addresses the following competency questions:
\begin{itemize}
    \item Which patients are currently in a critical condition?
    \item Does a specific surgery meet the mandatory staffing requirements (Surgeons, Anesthesiologists, and Nurses)?
    \item Are clinical resource allocations (e.g., ICU beds) consistent with patient assignments?
\end{itemize}

% -------------------------------------------------------------------
% Section 2: Ontology Architecture
% -------------------------------------------------------------------
\section{Ontology Architecture}
The ontology follows a top-down taxonomic structure rooted in \texttt{owl:Thing}, designed to ensure high dependency complexity (depth $>$ 5).

\subsection{Class Hierarchy}
The core hierarchy distinguishes between actors, events, and assets:
\begin{itemize}
    \item \textbf{Person}: The superclass for all human actors.
    \begin{itemize}
        \item \textbf{Patient}: Further specialized into \texttt{Inpatient} and \texttt{Outpatient}[cite: 36, 40].
        \item \textbf{Staff}: Subdivided into \texttt{AdministrativeStaff} and \texttt{MedicalStaff}[cite: 39].
    \end{itemize}
    \item \textbf{ClinicalEvent}: Covers temporal occurrences such as \texttt{Encounter} (visits) and \texttt{MedicalProcedure} (surgeries, diagnostic tests)[cite: 23].
    \item \textbf{Organization}: Includes \texttt{Hospital} and various \texttt{Department} subclasses like \texttt{EmergencyRoom} and \texttt{IntensiveCareUnit}[cite: 27].
\end{itemize}

\subsection{Disjointness Axioms}
To prevent logical errors (e.g., a person being both a Doctor and a Nurse), strict disjointness axioms were applied:
\begin{itemize}
    \item \texttt{Doctor} $\sqcap$ \texttt{Nurse} $\equiv \bot$ [cite: 31]
    \item \texttt{Inpatient} $\sqcap$ \texttt{Outpatient} $\equiv \bot$ [cite: 37]
    \item \texttt{DiagnosticTest} subclasses (MRI, X-Ray, BloodTest) are mutually disjoint[cite: 68].
\end{itemize}

% -------------------------------------------------------------------
% Section 3: Property Definitions
% -------------------------------------------------------------------
\section{Property Definitions}
The semantic expressivity of the ontology relies on over 40 defined properties.

\subsection{Object Properties (Roles)}
\begin{table}[H]
\centering
\begin{tabular}{@{}llp{6cm}@{}}
\toprule
\textbf{Property} & \textbf{Characteristics} & \textbf{Description} \\ \midrule
\texttt{prescribes} & Domain: Doctor & Links a doctor to a prescription[cite: 9]. \\
\texttt{performedBy} & Domain: Procedure & Crucial for validating staffing constraints[cite: 8]. \\
\texttt{supervises} & Transitive & Models hierarchical staff management[cite: 10]. \\
\texttt{admittedTo} & Domain: Patient & Links patients to departments[cite: 2]. \\
\bottomrule
\end{tabular}
\caption{Key Object Properties}
\end{table}

\subsection{Data Properties (Attributes)}
Attributes capture literal data values necessary for computation:
\begin{itemize}
    \item \textbf{\texttt{surgeryDurationMinutes}}: Used to track procedure length[cite: 19].
    \item \textbf{\texttt{isCritical}}: A Boolean flag driving inference rules[cite: 17].
    \item \textbf{\texttt{bedCount}}: Defines capacity for departments[cite: 13].
\end{itemize}

% -------------------------------------------------------------------
% Section 4: Logic and Constraints
% -------------------------------------------------------------------
\section{Logic and Reasoning}
This section details the Description Logic (DL) axioms that enforce domain rules.

\subsection{The "Two-Nurse" Rule}
A significant constraint was added to the \texttt{Surgery} class to ensure safety standards. The axiom states that every surgery must be performed by at least two nurses, in addition to a surgeon and an anesthesiologist.

\begin{lstlisting}[language=xml, caption={OWL Restriction for Surgery Staffing}]
<owl:Class rdf:about="...#Surgery">
    <rdfs:subClassOf>
        <owl:Restriction>
            <owl:onProperty rdf:resource="...#performedBy"/>
            <owl:minQualifiedCardinality rdf:datatype="&xsd;nonNegativeInteger">2</owl:minQualifiedCardinality>
            <owl:onClass rdf:resource="...#Nurse"/>
        </owl:Restriction>
    </rdfs:subClassOf>
</owl:Class>
\end{lstlisting}
\noindent \textit{Source: Ontology File, Lines 51-52}.

\subsection{Critical Patient Inference}
We defined \texttt{CriticalPatient} as an \textit{Equivalent Class} rather than a manual category. The reasoner automatically classifies any individual meeting these criteria:
\[ Patient \equiv Person \sqcap \exists admittedTo.IntensiveCareUnit \sqcap \exists isCritical.\{true\} \]
This ensures that any patient in the ICU flagged as critical is immediately recognized as a high-priority case [cite: 24-26].

% -------------------------------------------------------------------
% Section 5: Validation and Testing
% -------------------------------------------------------------------
\section{Validation and Testing}
The ontology was validated using the HermiT reasoner. We instantiated specific scenarios to test the Open World Assumption (OWA) and closure axioms.

\subsection{Scenario A: Compliant Surgery (Brain\_Surgery\_X99)}
\textbf{Setup:} An instance of \texttt{Surgery} was created with the following \texttt{performedBy} assertions:
\begin{itemize}
    \item \texttt{Dr\_Smith} (Surgeon) [cite: 57]
    \item \texttt{Dr\_Jones} (Anesthesiologist) [cite: 56]
    \item \texttt{Nurse\_Karen} (Nurse) [cite: 57]
    \item \texttt{Nurse\_Matteo} (Nurse) [cite: 58]
\end{itemize}
\textbf{Result:} The reasoner validated this instance as \textbf{Consistent}. The presence of two distinct nurses (\texttt{Nurse\_Karen} and \texttt{Nurse\_Matteo}) satisfied the `min 2 Nurse` cardinality constraint.

\subsection{Scenario B: Non-Compliant Surgery (Heart\_Transplant\_001)}
\textbf{Setup:} This instance was asserted with a \textit{Closure Axiom} stating that \textbf{only} the following staff were present:
\begin{itemize}
    \item \texttt{Dr\_Smith} (Surgeon)
    \item \texttt{Dr\_Jones} (Anesthesiologist)
\end{itemize}
\textbf{Logic Trace:}
\begin{enumerate}
    \item The class \texttt{Surgery} requires min 2 \texttt{Nurse} instances.
    \item The individual \texttt{Heart\_Transplant\_001} explicitly lists only Doctors in its closure axiom [cite: 60-61].
    \item The disjointness axiom (`Doctor` vs `Nurse`) confirms neither Smith nor Jones is a Nurse[cite: 31].
\end{enumerate}
\textbf{Result:} The reasoner successfully triggered an \textbf{Inconsistency Error}, proving the validation logic is functioning correctly.

\subsection{Scenario C: Critical Inference (John\_Doe)}
\textbf{Setup:} \texttt{John\_Doe} was asserted as a Patient admitted to \texttt{ICU\_Ward\_1} with \texttt{isCritical = true}[cite: 63].
\textbf{Result:} Upon synchronization, the reasoner inferred \texttt{John\_Doe} to be of type \texttt{CriticalPatient}, validating the defined class logic.

% -------------------------------------------------------------------
% Section 6: Conclusion
% -------------------------------------------------------------------
\section{Conclusion}
The developed Clinical Healthcare Ontology successfully meets the complexity requirements of the project. It features a deep taxonomy, over 40 semantic properties, and rigorous logic constraints including cardinality, disjointness, and property transitivity. The successful validation of the "Two-Nurse Rule" and critical patient inference demonstrates the ontology's utility in automated clinical decision support.

\end{document}