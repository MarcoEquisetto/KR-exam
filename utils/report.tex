\documentclass[11pt,a4paper]{article}

\usepackage[margin=2.5cm]{geometry}
\usepackage{graphicx}
\usepackage{amsmath,amssymb}
\usepackage{booktabs}
\usepackage{caption}
\usepackage{listings}
\usepackage{xcolor}
\usepackage{microtype}
\usepackage{hyperref}
\usepackage[parfill]{parskip}
\usepackage{float}

% Hyperlink configuration
\hypersetup{
    colorlinks=true,
    linkcolor=blue,
    citecolor=blue,
    urlcolor=blue,
    pdftitle={Clinical Healthcare Ontology},
    pdfauthor={Marco Equisetto}
}

% Listings configuration (Adapted for XML/OWL)
\lstset{
    basicstyle=\ttfamily\small,
    keywordstyle=\color{blue},
    commentstyle=\color{gray},
    stringstyle=\color{red},
    breaklines=true,
    numbers=left,
    numberstyle=\tiny,
    frame=single,
    tabsize=4,
    captionpos=b,
    language=XML,
    morekeywords={owl:Class, owl:Restriction, owl:onProperty, owl:minQualifiedCardinality, owl:onClass, owl:equivalentClass, owl:intersectionOf, rdf:Description, rdfs:subClassOf, owl:allValuesFrom, owl:someValuesFrom, owl:hasValue}
}

\title{\textbf{Clinical Healthcare Ontology} \\
       \large Artificial Intelligence Master Degree \\
       \textbf{Knowledge Representation}}
\author{Marco Equisetto \\ University of Verona}
\date{\today}

\begin{document}

\maketitle

\begin{abstract}
This report documents the design, implementation, and logical validation of a semantic ontology for the Clinical Healthcare domain. Developed using \textbf{OWL 2 DL}, the system models complex hospital environments, including staff hierarchies, patient care workflows, pharmaceutical management, and procedural safety protocols. Key features include the automated inference of patient risk status (e.g., \texttt{CriticalPatient}), hierarchical staffing constraints (e.g., \texttt{HeadNurse}, \texttt{SeniorSurgeon}), and strict safety protocols distinguishing between \texttt{Surgery} and \texttt{Vaccination} workflows. The ontology utilizes over 40 semantic properties, includes drug allergy tracking, and has been validated using the HermiT reasoner.
\end{abstract}

\tableofcontents
\newpage

\section{Introduction}
This project aims to create a formal ontology that acts not merely as a database schema, but as an intelligent logic structure capable of automated inference. The ontology addresses the following \textbf{Competency Questions}:

\begin{enumerate}
    \item \textbf{Safety Compliance:} Does a specific surgery meet mandatory staffing requirements (e.g., presence of a Surgeon, Anesthesiologist, and at least two Nurses)?
    \item \textbf{Risk Stratification:} Which patients are currently in a "Critical" state based on their department admission and vitals?
    \item \textbf{Pharmaceutical Safety:} Is a patient prescribed drugs to which they have a recorded allergy?
    \item \textbf{Resource Verification:} Are medical procedures assigned to the correct staff specializations (e.g., Nurses for vaccinations vs. multi-disciplinary teams for surgery)?
\end{enumerate}

\section{Ontology Structure (Taxonomy)}
The ontology follows a top-down taxonomic structure rooted in \texttt{owl:Thing}. It features a hierarchy depth greater than 5 levels, ensuring high dependency complexity.

\subsection{Hierarchy}
The class structure differentiates between \texttt{Person} (subdivided into \texttt{Staff} and \texttt{Patient}), \texttt{ClinicalEvent} (including \texttt{Surgery} and \texttt{Vaccination}), and \texttt{MedicalAsset} (including \texttt{Pharmaceuticals} and \texttt{Equipment}).

\begin{figure}[H]
    \centering
    % Placeholder for image file as requested
    \includegraphics[width=0.7\textwidth]{higherarchy.png}
    \caption{Hierarchy of Taxonomy}
    \label{fig:hierarchy}
\end{figure}

\subsection{Disjointness Axioms}
To prevent logical errors, strict disjointness axioms were applied:
\begin{itemize}
    \item \textbf{Roles:} The class \texttt{Doctor} is disjoint from \texttt{Nurse}. Within the \texttt{Doctor} class, \texttt{Anesthesiologist}, \texttt{GeneralPractitioner}, and \texttt{Surgeon} are mutually disjoint.
    \item \textbf{Patients:} \texttt{Inpatient} and \texttt{Outpatient} are mutually exclusive.
    \item \textbf{Pharmaceuticals:} \texttt{Antibiotic}, \texttt{Analgesic}, and \texttt{Vitamin} are defined as disjoint classes to prevent misclassification of drugs.
\end{itemize}

\section{Property Definitions}
The ontology employs a robust set of properties to define relationships and attributes.

\subsection{Object Properties (Roles)}
Object properties define relationships between class instances.

\begin{table}[H]
\centering
\begin{tabular}{@{}llp{6cm}@{}}
\toprule
\textbf{Property} & \textbf{Characteristics} & \textbf{Description} \\ \midrule
\texttt{performedBy} & Domain: Procedure & Links procedures to staff; crucial for validation rules. \\
\texttt{supervises} & Transitive & Models hierarchical staff management (e.g., \texttt{HeadNurse} supervises \texttt{Nurse}). \\
\texttt{allergicTo} & Domain: Patient & Links a patient to a specific \texttt{Drug} instance (e.g., Penicillin). \\
\texttt{admittedTo} & Domain: Patient & Specifies the department (e.g., ICU) where a patient resides. \\
\texttt{usesEquipment} & Domain: Procedure & Links procedures to required tools (e.g., \texttt{SurgicalTool}). \\
\bottomrule
\end{tabular}
\caption{Key Object Properties}
\end{table}

\subsection{Data Properties (Attributes)}
Attributes capture literal values used for logical computation:
\begin{itemize}
    \item \textbf{\texttt{surgeryDurationMinutes}}: Used to infer if a surgery is "High Risk".
    \item \textbf{\texttt{isCritical}}: A Boolean flag driving patient status inference.
    \item \textbf{\texttt{yearsOfExperience}}: Integer used to classify senior staff.
    \item \textbf{\texttt{bedCount}}: Defines capacity for departments like the ICU.
\end{itemize}

\section{Logical Constraints \& Reasoning}
This section details the Description Logic (DL) axioms that provide the ontology with its "intelligence."

\subsection{Procedural Staffing Constraints}
We implemented distinct cardinality constraints for different medical procedures:

\newpage

\subsubsection{The "Two-Nurse" Rule (Surgery)}
A strict constraint was added to the \texttt{Surgery} class. The axiom mandates that every surgery must be performed by \textbf{at least two} nurses, in addition to a surgeon and an anesthesiologist.

\begin{lstlisting}[caption={OWL Restriction for Surgery Staffing}]
<owl:Class rdf:about="...#Surgery">
    <rdfs:subClassOf>
        <owl:Restriction>
            <owl:onProperty rdf:resource="...#performedBy"/>
            <owl:minQualifiedCardinality rdf:datatype="&xsd;nonNegativeInteger">2</owl:minQualifiedCardinality>
            <owl:onClass rdf:resource="...#Nurse"/>
        </owl:Restriction>
    </rdfs:subClassOf>
</owl:Class>
\end{lstlisting}

\subsubsection{Vaccination Protocols}
Unlike surgery, the \texttt{Vaccination} class is defined with a lighter constraint, requiring \textbf{at least one} nurse and the use of specific \texttt{SurgicalTool} equipment.

\subsection{Staff Hierarchy (Head Nurse)}
The \texttt{HeadNurse} class is defined not just as a subclass of \texttt{Nurse}, but through a property restriction:
\[ \text{HeadNurse} \equiv \text{Nurse} \sqcap \exists \text{supervises.Nurse} \]
This ensures that any nurse supervising another nurse is automatically classified as a Head Nurse.

\subsection{Automated Inference (Defined Classes)}
We implemented \texttt{owl:equivalentClass} definitions to allow the reasoner to automatically classify individuals based on data patterns.

\begin{enumerate}
    \item \textbf{CriticalPatient}: The reasoner automatically classifies a patient as "Critical" if they are admitted to the ICU \textbf{AND} have the \texttt{isCritical} flag set to true.
    \[ \text{Patient} \sqcap \exists \text{admittedTo.ICU} \sqcap \exists \text{isCritical.\{true\}} \]

    \item \textbf{SeniorSurgeon}: Any surgeon with 15 or more years of experience is automatically reclassified as a Senior Surgeon.
    \[ \text{Surgeon} \sqcap \text{yearsOfExperience} \ge 15 \]

    \item \textbf{HighRiskSurgery}: Any surgery lasting longer than 300 minutes is flagged as High Risk.
\end{enumerate}

\section{Validation and Testing}
The ontology was populated with diverse individuals (ABox) to test the Open World Assumption (OWA) and strict validation rules.

\subsection{Scenario A: Compliant Surgery Execution}
\textbf{Instance:} \texttt{Brain\_Surgery\_X99} \\
\textbf{Setup:} This surgery is explicitly linked to:
\begin{itemize}
    \item \texttt{Dr\_Smith} (Surgeon) and \texttt{Dr\_Jones} (Anesthesiologist)
    \item \texttt{Nurse\_Karen} and \texttt{Nurse\_Matteo}
    \item Equipment: \texttt{Surgical\_Laser\_1}
\end{itemize}
\textbf{Result:} The reasoner validated this instance as \textbf{Consistent}. The presence of two distinct nurses and the requisite medical staff satisfied the complex cardinality constraints of the \texttt{Surgery} class.

\subsection{Scenario B: Complex Staffing \& Hierarchy Check}
\textbf{Instance:} \texttt{Heart\_Transplant\_001} \\
\textbf{Setup:} This high-complexity instance involves \texttt{Dr\_Smith}, \texttt{Dr\_Jones}, \texttt{Nurse\_Betty}, and \texttt{Nurse\_Linda}. \\
\textbf{Inference:} Additionally, \texttt{Nurse\_Ratched} is defined as supervising \texttt{Nurse\_Betty}. The reasoner successfully infers \texttt{Nurse\_Ratched} as type \texttt{HeadNurse} due to the existence of the `supervises` relationship, validating the hierarchical logic without manual type assertion.

\subsection{Scenario C: Pharmaceutical Allergy Tracking}
\textbf{Instance:} \texttt{Patient\_Zero} \\
\textbf{Setup:} Defined as an \texttt{Outpatient} with the property \texttt{allergicTo} pointing to \texttt{Penicillin\_V}. \\
\textbf{Logic Trace:}
\begin{enumerate}
    \item \texttt{Penicillin\_V} is classified as an \texttt{Antibiotic}.
    \item The ontology successfully links the patient to the allergen.
    \item This structure enables safety queries (e.g., "Find all patients allergic to Antibiotics") to return \texttt{Patient\_Zero} correctly.
\end{enumerate}

\subsection{Scenario D: Inference Logic & Risk Classification}
\begin{itemize}
    \item \textbf{Seniority Test:} \texttt{Dr\_House} (Experience: 20 years) was correctly inferred as type \texttt{SeniorSurgeon}.
    \item \textbf{Pediatric Test:} \texttt{Patient\_Timmy} (Age: 8) was correctly inferred as type \texttt{PediatricPatient}.
    \item \textbf{High Risk Surgery:} \texttt{Complex\_Bypass\_001} (Duration: 450 min) was correctly inferred as type \texttt{HighRiskSurgery}.
    \item \textbf{Vaccination Check:} \texttt{School\_Vax\_03} was validated as a consistent \texttt{Vaccination} event, performed by \texttt{Nurse\_Linda} on \texttt{Patient\_Timmy}.
\end{itemize}

\section{Conclusion}
The developed Clinical Healthcare Ontology successfully meets the project complexity requirements. It features a deep taxonomy distinguishing between various medical events and assets, over 40 semantic properties, and rigorous logic constraints.

The successful validation of the "Two-Nurse Rule" for surgeries, the distinct protocol for vaccinations, and the automated handling of staff hierarchies (Head Nurse) and patient risks demonstrate the ontology's capability to support automated clinical decision-making.

\end{document}