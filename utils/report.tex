\documentclass[11pt,a4paper]{article}

\usepackage[margin=2.5cm]{geometry}
\usepackage{graphicx}
\usepackage{amsmath,amssymb}
\usepackage{booktabs}
\usepackage{caption}
\usepackage{listings}
\usepackage{xcolor}
\usepackage{microtype}
\usepackage{hyperref}
\usepackage[parfill]{parskip}
\usepackage{float}

% Hyperlink configuration
\hypersetup{
    colorlinks=true,
    linkcolor=blue,
    citecolor=blue,
    urlcolor=blue,
    pdftitle={Clinical Healthcare Ontology},
    pdfauthor={Marco Equisetto}
}

% Listings configuration (Adapted for XML/OWL)
\lstset{
    basicstyle=\ttfamily\small,
    keywordstyle=\color{blue},
    commentstyle=\color{gray},
    stringstyle=\color{red},
    breaklines=true,
    numbers=left,
    numberstyle=\tiny,
    frame=single,
    tabsize=4,
    captionpos=b,
    language=XML,
    morekeywords={owl:Class, owl:Restriction, owl:onProperty, owl:minQualifiedCardinality, owl:onClass, owl:equivalentClass, owl:intersectionOf, rdf:Description, rdfs:subClassOf}
}

\title{\textbf{Clinical Healthcare Ontology} \\
       \large Artificial Intelligence Master Degree \\
       \textbf{Knowledge Representation}}
\author{Marco Equisetto \\ University of Verona}
\date{\today}

\begin{document}

\maketitle

\begin{abstract}
This report documents the design, implementation, and logical validation of a semantic ontology for the Clinical Healthcare domain. Developed using \textbf{OWL 2 DL}, the system models hospital environments, including staff hierarchies, patient care workflows, and surgical requirements.

Key features include the automated inference of patient risk status (e.g., \texttt{CriticalPatient}), hierarchical staffing constraints (e.g., \texttt{SeniorSurgeon}), and strict safety protocols such as the "Two-Nurse Rule" for surgeries. The ontology utilizes over 40 semantic properties and has been validated using the HermiT reasoner.
\end{abstract}

\tableofcontents
\newpage

\section{Introduction}
This project aims to create a formal ontology that acts not merely as a database schema, but as an intelligent logic structure that can be then given to an AI model and inferred upon.

The ontology addresses the following \textbf{Competency Questions}:
\begin{enumerate}
    \item \textbf{Safety Compliance:} Does a specific surgery meet mandatory staffing requirements (e.g., presence of a Surgeon, Anesthesiologist, and at least two Nurses)?
    \item \textbf{Risk Stratification:} Which patients are currently in a "Critical" state based on their department admission and vitals?
    \item \textbf{Resource Verification:} Are medical procedures assigned to the correct staff specializations?
\end{enumerate}

\section{Ontology Structure (Taxonomy)}
The ontology follows a top-down taxonomic structure rooted in \texttt{owl:Thing}. It features a hierarchy depth greater than 5 levels, ensuring high dependency complexity.

\subsection{Hierarchy}
\begin{figure}[H]
    \centering
    % Placeholder for image file
    \includegraphics[width=0.7\textwidth]{higherarchy.png}
    \caption{Hierarchy of Taxonomy}
    \label{fig:hierarchy}
\end{figure}

\subsection{Disjointness Axioms}
To prevent logical errors, strict disjointness axioms were applied. For example, the class \texttt{Doctor} is disjoint from \texttt{Nurse}, ensuring no individual can hold both roles simultaneously. Similarly, \texttt{Inpatient} and \texttt{Outpatient} are mutually exclusive.

\section{Property Definitions}
The ontology employs over 40 properties to define relationships and attributes.

\subsection{Object Properties (Roles)}
Object properties define relationships between class instances.

\begin{table}[H]
\centering
\begin{tabular}{@{}llp{6cm}@{}}
\toprule
\textbf{Property} & \textbf{Characteristics} & \textbf{Description} \\ \midrule
\texttt{performedBy} & Domain: Procedure & Links procedures to staff; crucial for validation rules. \\
\texttt{supervises} & Transitive & Models hierarchical staff management (A supervises B, B supervises C $\rightarrow$ A supervises C). \\
\texttt{treats} & Inverse (isTreatedBy) & Connects \texttt{MedicalStaff} to \texttt{Patient}. \\
\texttt{admittedTo} & Domain: Patient & Specifies the department (e.g., ICU) where a patient resides. \\
\bottomrule
\end{tabular}
\caption{Key Object Properties}
\end{table}

\subsection{Data Properties (Attributes)}
Attributes capture literal values used for logical computation:
\begin{itemize}
    \item \textbf{\texttt{surgeryDurationMinutes}}: Used to infer if a surgery is "High Risk".
    \item \textbf{\texttt{isCritical}}: A Boolean flag driving patient status inference.
    \item \textbf{\texttt{yearsOfExperience}}: Integer used to classify senior staff.
\end{itemize}

\section{Logical Constraints \& Reasoning}
This section details the Description Logic (DL) axioms that provide the ontology with its "intelligence."

\subsection{The "Two-Nurse" Rule (Safety Constraint)}
A strict constraint was added to the \texttt{Surgery} class. The axiom mandates that every surgery must be performed by \textbf{at least two} nurses, in addition to a surgeon and anesthesiologist.

\begin{lstlisting}[caption={OWL Restriction for Surgery Staffing}]
<owl:Class rdf:about="...#Surgery">
    <rdfs:subClassOf>
        <owl:Restriction>
            <owl:onProperty rdf:resource="...#performedBy"/>
            <owl:minQualifiedCardinality rdf:datatype="&xsd;nonNegativeInteger">2</owl:minQualifiedCardinality>
            <owl:onClass rdf:resource="...#Nurse"/>
        </owl:Restriction>
    </rdfs:subClassOf>
</owl:Class>
\end{lstlisting}
\textit{Source: Ontology File, Lines 68-69.}

\subsection{Automated Inference (Defined Classes)}
We implemented \texttt{owl:equivalentClass} definitions to allow the reasoner to automatically classify individuals based on data patterns.

\begin{enumerate}
    \item \textbf{CriticalPatient}: The reasoner automatically classifies a patient as "Critical" if they are admitted to the ICU \textbf{AND} have the \texttt{isCritical} flag set to true.
    \[ \text{Patient} \sqcap \exists \text{admittedTo.ICU} \sqcap \exists \text{isCritical.\{true\}} \]

    \item \textbf{SeniorSurgeon}: Any surgeon with 15 or more years of experience is automatically reclassified as a Senior Surgeon.
    \[ \text{Surgeon} \sqcap \text{yearsOfExperience} \ge 15 \]

    \item \textbf{HighRiskSurgery}: Any surgery lasting longer than 300 minutes is flagged as High Risk.
\end{enumerate}

\section{Validation and Testing}
The ontology was populated with diverse individuals to test the Open World Assumption (OWA) and strict validation rules.

\subsection{Scenario A: Compliant Surgery}
\textbf{Instance:} \texttt{Brain\_Surgery\_X99} \\
\textbf{Setup:} Linked to \texttt{Dr\_Smith} (Surgeon), \texttt{Dr\_Jones} (Anesthesiologist), \texttt{Nurse\_Karen}, and \texttt{Nurse\_Matteo}. \\
\textbf{Result:} The reasoner validated this instance as \textbf{Consistent}. The presence of two distinct nurses satisfied the \texttt{min 2 Nurse} cardinality constraint.

\subsection{Scenario B: Non-Compliant Surgery Check}
\textbf{Instance:} \texttt{Heart\_Transplant\_001} \\
\textbf{Setup:} Asserted with a \textit{Closure Axiom} stating \textbf{only} \texttt{Dr\_Smith} and \texttt{Dr\_Jones} were present. \\
\textbf{Logic Trace:}
\begin{enumerate}
    \item The class \texttt{Surgery} requires min 2 \texttt{Nurse} instances.
    \item The individual lists only Doctors in its closure axiom.
    \item \textbf{Result:} The reasoner triggered an \textbf{Inconsistency Error}, successfully verifying the safety rule.
\end{enumerate}

\subsection{Scenario C: Inference Logic}
\begin{itemize}
    \item \textbf{Seniority Test:} \texttt{Dr\_House} (Experience: 20) was correctly inferred as type \texttt{SeniorSurgeon}.
    \item \textbf{Pediatric Test:} \texttt{Patient\_Timmy} (Age: 8) was correctly inferred as type \texttt{PediatricPatient}.
    \item \textbf{Risk Test:} \texttt{Complex\_Bypass\_001} (Duration: 450 min) was correctly inferred as type \texttt{HighRiskSurgery}.
\end{itemize}

\section{Conclusion}
The developed Clinical Healthcare Ontology successfully meets the project complexity requirements. It features a deep taxonomy, over 40 semantic properties, and rigorous logic constraints including cardinality, disjointness, and property transitivity. The successful validation of the "Two-Nurse Rule" and the automated inference of patient risk demonstrate the ontology's utility in automated clinical decision support systems.

\end{document}